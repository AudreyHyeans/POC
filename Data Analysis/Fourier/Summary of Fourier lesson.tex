\documentclass{article}
\usepackage[utf8]{inputenc}
\usepackage[english]{babel}
\usepackage{amssymb}
\usepackage{graphicx}
\usepackage{listings}
\usepackage{listing}
\title{Summary for the Fourier lesson }
\begin{document}
\section{Formulas}
\subsection{Definition of a signal}
A signal is a record of a process that varies either in time or in space. It's a sum of various periodically quantities. \newline
The aim of this lesson is to decompose a signal into a sum of cosinus and sinus.

\subsubsection{Fourier series : principle}
Let's consider a periodic fonction $h(t)$ such that $h(t) = h(t+T)$ with T the temporal period. \newline

We want to decompose h into a sum of cosinus and sinus : \newline

$h(t)= \sum_{n=0}^{\infty} An*sin(2*\pi*f_n*t+\phi_n)$\newline
After some calculus that you can check in the demonstration 1. \newline
$h(t) = \frac{a_0}{2} + \sum_{n=1}^{\infty} a_n*\sin(2*\pi*f_n*t) + b_n*\cos(2*\pi*f_n*t)$ \newline
with $a_n = \frac{2}{T} * \int_{0}^{T} h(t)*\sin(2*\pi*f_n*t)dt$ , $b_n = \frac{2}{T} * \int_{0}^{T} h(t)*\cos(2*\pi*f_n*t)dt$ and $a_0 =\frac{2}{T} * \int_{0}^{T} h(t)dt$


\section{Formulas and demonstrations}
\end{document}